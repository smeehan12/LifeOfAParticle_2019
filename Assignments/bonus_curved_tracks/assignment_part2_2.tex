\documentclass[12pt]{article}
%	options include 12pt or 11pt or 10pt
%	classes include article, report, book, letter, thesis

% \usepackage[margin=0.75in]{geometry}
\usepackage[margin=1in]{geometry}
\setlength{\parindent}{0pt}

\usepackage{hyperref}

%for writing of code in blocks like
%\begin{lstlisting}
%   .......
%\end{lstlisting}
\usepackage{listings}
\usepackage{color}
\usepackage{enumitem}
\usepackage{graphicx}
\usepackage{url}

\definecolor{dkgreen}{rgb}{0,0.6,0}
\definecolor{gray}{rgb}{0.5,0.5,0.5}
\definecolor{mauve}{rgb}{0.58,0,0.82}

\lstset{frame=tb,
  language=C++,
  aboveskip=3mm,
  belowskip=3mm,
  showstringspaces=false,
  columns=flexible,
  basicstyle={\small\ttfamily},
  numbers=none,
  numberstyle=\tiny\color{gray},
  keywordstyle=\color{blue},
  commentstyle=\color{dkgreen},
  stringstyle=\color{mauve},
  breaklines=true,
  breakatwhitespace=true,
  tabsize=3
}
%%%%%%%%%%%%%%%%%%%%%%

\title{Life of a Particle : Assignment 2}
\author{Claire David}
\date{Due Date : DD/MM/2019}

\begin{document}
\vspace{-2ex}
\maketitle

\textbf{How to submit}
\newline
This assignment should be submitted by replying to the email sent out requesting its submission.\\
You should include a single PDF file that has any verbal description of the answers to the questions, along with description of what computer code files go with which question.\\
Please bundle this all into a single tarball and submit this one tarball file.\\
Naming convention: {\tt{lifefofparticle\_part2\_assignment1\_YOURNAME.tar}}

\section*{Particle separation through measurement of energy loss}

\textit{Have this on my German lectures, will translate and write it here.}

\section*{Curved track fitting in constant magnetic field}

Will write properly. Idea is:

\begin{itemize}
 \item In the toy detector, I provide measurements on each layer of the Y positions of a 'pixel-hit', along with its measurement error. They draw the points. See it makes sense with presence of B field.

 \item They fit the track. Get fit parameters.

 \item They compute the resolution on the track curvature.

 \item Case 1: the pixel size is reduced by factor 2. Fancier detector. Refitting: how is the resolution improved? And for a reduction of pixel side by factor 3?

 \item Case 2: back to regular pixel width. But an extra layer is added at the end of the detector. The fit has one extra point. How is the resolution improved?

\end{itemize}



\end{document}