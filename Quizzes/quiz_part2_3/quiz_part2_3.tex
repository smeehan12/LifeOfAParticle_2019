\documentclass[12pt]{article}
%	options include 12pt or 11pt or 10pt
%	classes include article, report, book, letter, thesis

% \usepackage[margin=0.75in]{geometry}
\usepackage[margin=1in]{geometry}
\setlength{\parindent}{0pt}

\usepackage{hyperref}

%for writing of code in blocks like
%\begin{lstlisting}
%   .......
%\end{lstlisting}
\usepackage{listings}
\usepackage{color}
\usepackage{enumitem}
\usepackage{graphicx}

\definecolor{dkgreen}{rgb}{0,0.6,0}
\definecolor{gray}{rgb}{0.5,0.5,0.5}
\definecolor{mauve}{rgb}{0.58,0,0.82}

\lstset{frame=tb,
  language=C++,
  aboveskip=3mm,
  belowskip=3mm,
  showstringspaces=false,
  columns=flexible,
  basicstyle={\small\ttfamily},
  numbers=none,
  numberstyle=\tiny\color{gray},
  keywordstyle=\color{blue},
  commentstyle=\color{dkgreen},
  stringstyle=\color{mauve},
  breaklines=true,
  breakatwhitespace=true,
  tabsize=3
}
%%%%%%%%%%%%%%%%%%%%%%

\title{Life of a Particle : Quiz on Semi-conductor Physics}
\author{Claire David}
\date{Due Date : DD/MM/2019}

\begin{document}
\maketitle

\section{Conductivity of intrinsic silicon}
Calculate the voltage that should be applied on a rectangular plate of intrinsic (undoped) silicon at 300 K in order to have a current of 100 nA.\\
The cross section of the plate is 10 $\mu m$ $\times$ 50 $\mu m$ and its length is 1 mm.

\section{Energy bands and acceptors}
A) Sketch the band structure of an intrinsic, a $p$-doped and $n$-doped semi-conductor. Indicate for each the Fermi level.\\

B) Why is this doping $n$ and $p$ called donor and acceptors respectively?\\

C) A $p$-doped silicon plate is put next to an $n$-doped one. Sketch the density of charges, the electric field, the electrostatic potential and the band structure with the Fermi level as a function of the x coordinate:
\begin{enumerate}
 \item when the two stabs are apart from each other
 \item when the stabs touch
\end{enumerate}

\section{Conductivity of doped silicon}
The silicon plate of exercice 1 is now doped with donors of different densities:
\begin{enumerate}
 \item $N_D = 10^9$ cm$^{-3}$
 \item $N_D = 10^12$ cm$^{-3}$
 \item $N_D = 10^15$ cm$^{-3}$
\end{enumerate}

Which voltages must be applied to achieve the same current?






    
\end{document}