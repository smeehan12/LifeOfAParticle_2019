\documentclass[12pt]{article}
%	options include 12pt or 11pt or 10pt
%	classes include article, report, book, letter, thesis

% \usepackage[margin=0.75in]{geometry}
\usepackage[margin=1in]{geometry}
\setlength{\parindent}{0pt}

\usepackage{hyperref}

%for writing of code in blocks like
%\begin{lstlisting}
%   .......
%\end{lstlisting}
\usepackage{listings}
\usepackage{color}
\usepackage{enumitem}
\usepackage{graphicx}

\definecolor{dkgreen}{rgb}{0,0.6,0}
\definecolor{gray}{rgb}{0.5,0.5,0.5}
\definecolor{mauve}{rgb}{0.58,0,0.82}

\lstset{frame=tb,
  language=C++,
  aboveskip=3mm,
  belowskip=3mm,
  showstringspaces=false,
  columns=flexible,
  basicstyle={\small\ttfamily},
  numbers=none,
  numberstyle=\tiny\color{gray},
  keywordstyle=\color{blue},
  commentstyle=\color{dkgreen},
  stringstyle=\color{mauve},
  breaklines=true,
  breakatwhitespace=true,
  tabsize=3
}
%%%%%%%%%%%%%%%%%%%%%%

\title{Life of a Particle : Passage of Particles Through Matter}
\author{Claire David}
\date{Due Date : DD/MM/2019}

\begin{document}
\maketitle


\section{Why do we see muons on Earth?}

Cosmic rays colliding with air in the upper atmosphere (10,000 km) produce muons. The lifetime of the muon is $\tau_\mu = 2.2 \times 10^{-6}$ s.\\
A rough estimation on their average distance would be in the order of 600 m.\\

\textbf{Question A:} Accounting for relativistic effects, what is the average distance a muon of kinetic energy $E_\mu = 2$ GeV will cover?\\

\textbf{Question B:} What would be this distance for pions?\\


\vspace{4ex}
\hrule
\vspace{4ex}
\textit{Data:}\\


The total kinetic energy of a particle of rest mass $m_0$ and velocity $v$ is given by:
\begin{equation}
 E = mc^2 = \sqrt{pc^2 + m_0^2 \: c^4}
\end{equation}
with $p$ the momentum of the particle, $p= m_0 v \gamma$\\

The relativistic factor $\gamma$ is defined by:
\begin{equation}
 \gamma = \frac{1}{\sqrt{1 - \frac{v^2}{c^2}}}
\end{equation}
where $c$ is the speed of light.\\

The muon has a rest mass of $m_\mu$ = 105 MeV.

\vspace{4ex}
\hrule
\vspace{4ex}


\section{Stopping power}

Calculate the stopping power of 5 MeV $\alpha$-particles in air.

\end{document}
